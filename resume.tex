\documentclass[12pt,a4paper,roman]{moderncv}   % possible options include font size ('10pt', '11pt' and '12pt'), paper size ('a4paper', 'letterpaper', 'a5paper', 'legalpaper', 'executivepaper' and 'landscape') and font family ('sans' and 'roman')

% moderncv themes
\moderncvstyle{classic}                        % style options are 'casual' (default) and 'classic' 
\moderncvcolor{blue}                          % color options 'blue' (default), 'orange', 'green', 'red', 'purple', 'grey' and 'black'
%\renewcommand{\familydefault}{\rmdefault}    % to set the default font; use '\sfdefault' for the default sans serif font, '\rmdefault' for the default roman one, or any tex font name
\usepackage{textcomp}

%\nopagenumbers{}                             % uncomment to suppress automatic page numbering for CVs longer than one page

% character encoding
\usepackage[utf8]{inputenc}                   % replace by the encoding you are using
%\usepackage{CJKutf8}                         % if you need to use CJK to typeset your resume in Chinese, Japanese or Korean

% adjust the page margins
\usepackage[scale=0.8]{geometry}
\setlength{\hintscolumnwidth}{3.075cm}           % if you want to change the width of the column with the dates

% personal data
\firstname{George}
\familyname{Shakhnazaryan}
\title{Software Engineer}               % optional, remove the line if not wanted
%\address{Lake Zurich, IL }{60047}    % optional, remove the line if not wanted
%\mobile{+1~(847)~555~5555}                     % optional, remove the line if not wanted
%\phone{+2~(345)~678~901}                      % optional, remove the line if not wanted
%\fax{+3~(456)~789~012}                        % optional, remove the line if not wanted
\email{george@shakhnazaryan.com}                          % optional, remove the line if not wanted
\homepage{https://github.com/gshakhn}                    % optional, remove the line if not wanted
%\extrainfo{additional information}            % optional, remove the line if not wanted
%\photo[64pt][0.4pt]{picture}                  % '64pt' is the height the picture must be resized to, 0.4pt is the thickness of the frame around it (put it to 0pt for no frame) and 'picture' is the name of the picture file; optional, remove the line if not wanted
%\quote{Objective: A technically challenging position in a team based environment.}                 % optional, remove the line if not wanted
	
\begin{document}
\maketitle

\section{Skills}
\cvitem{Core Languages}{Java, Scala, JavaScript}
\cvitem{Extra Languages}{Ruby, C\#, Lua, C}
\cvitem{Frameworks}{Stripes, Jersey, Scalatra, jQuery, Ember.js, Meteor, Guice, \mbox{Hibernate}, Slick}
\cvitem{TDD Tools}{JUnit, ScalaTest, ScalaCheck, Fitnesse, RSpec, Selenium WebDriver, \mbox{page-object}, Jasmine, \mbox{js-test-driver}}{
\cvitem{Quality Tools}{Sonar, FxCop, Scalastyle, scct, Jenkins, \mbox{CruiseControl.NET}}
\cvitem{Databases}{Oracle 11g, SQL Server 2008, HSQLDB 2.0, H2}
\cvitem{Source Control}{SVN, Git, Perforce}
\cvitem{Other}{JRebel, RabbitMQ, Ant, Ivy, sbt, maven, make}
%\cvitem{Operating Systems}{Linux, Windows, Mac OS X}
%\cvitem{Application Servers}{JBoss, Tomcat, IIS}
%\cvitem{IDEs}{IntelliJ, Visual Studio, Eclipse}
%\cvitem{Profilers}{YourKit, \mbox{EQATEC Profiler}}

\section{Experience}
\cventry{2010--Present}{Software Engineer}{Backstop Solutions Group}{Chicago, IL}{}{Software as a service platform for hedge funds and fund of funds.
\begin{itemize}
  \item Maintained and improved main Accounting/CRM Java web platform following Scrum and Kanban methodologies.
  \item Worked with product owners to write executable acceptance tests that caught numerous bugs during project development and subsequent maintenance.
  \item Collaborated with product owners to figure out project priorities.
  \item Trained new team members on platform and team processes.
  \item Deployed and monitored software in production.
  \item Developed separate accounting service to make business logic more traceable.
  \item Created scalable services to sync email from Exchange 2007/2010 and IMAP servers. The new services made syncing errors easier to diagnose.
  \item Integrated Fitnesse with build process. Promoted writing integration tests and assisted team members in test writing.
  \item Integrated js-test-driver with build process. This allowed us to test our JavaScript business logic separately from the web portion.
  \item Added new functionality and fixed bugs in page-object, an external Ruby gem that provides a  DSL for browser testing. page-object helped us write more maintainable and readable tests.
  \item Wrote and helped others write RSpec tests using Selenium WebDriver and page-object.
  \item Modified RSpec test runner to record videos of browser tests. This made diagnosing test failures easier.
  %\item Setup and maintained local Sonar installation for static analysis of codebase.
  \item Setup Vagrant recipe for development environment. This cut new developer setup time in half.
\end{itemize}
}
\cventry{2008-2010}{Software Developer}{AgileTek LLC}{Des Plaines, IL}{}{Custom software development firm specializing in high risk projects
\begin{itemize}
  \item Planned and estimated tasks, following an Agile methodology
  \item Created a .NET 3.5 WinForms application for an automated medical diagnostics device.
    \begin{itemize}
      \item Created a printing framework utilizing .NET printing to print reports. Designed and created reports using framework.
      \item Architected migration process of legacy data to new schema.
    \end{itemize}
  \item Created an online payment portal using ASP.NET Web Forms for auto \mbox{insurance} agencies, which thousands of customers use to pay for their auto insurance.
    \begin{itemize}
      \item Defined the client's objective and brainstormed ways of meeting the objective.
      \item Designed MS SQL Server database schema.
      \item Created FTP application to download payment records from a payment processor.
      \item Integrated multiple 3rd party software packages to ease client workflow.
      \item Setup SVN server and assisted team members in accessing source control.
    \end{itemize}
\end{itemize}}

\section{Education}
\cventry{2012}{Functional Programming Principles in Scala}{coursera.org}{}{}{An online course taught by Scala designer Martin Odersky}
\cventry{2012}{Artificial Intelligence for Robotics}{udacity.com}{}{}{An online course taught by Stanford professor Sebastian Thrun}
\cventry{2011}{Introduction to Artificial Intelligence}{ai-class.com}{}{}{An online course taught by Stanford professors Peter Norvig and Sebastian Thrun}
\cventry{2006--2008}{Bachelor of Science}{University of Illinois at Chicago}{}{\textit{3.66/4.00}}{Major: Computer Science}

\section{Personal Projects}
\cventry{}{sonar-intellij-plugin}{}{}{}{
  Plugin for IntelliJ IDE that connects to a Sonar server and shows code violations. Downloaded by 3000+ developers.
}
\cventry{}{idea-fitnesse}{}{}{}{
  Plugin for IntelliJ IDE that adds Fitnesse test support. Currently in development.
}
\cventry{}{Lending Shak}{}{}{}{
  Android app that connects to a Prosper account to show balances and notes.
}
\cventry{}{checkin}{}{}{}{
  Meteor app that helps team members share status updates.
}

%\section{Languages}
%\cvitem{English}{Fluent}
%\cvitem{Russian}{Spoken}

\end{document}
